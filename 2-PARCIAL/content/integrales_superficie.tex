\section{Integrales de superficie}

\subsection{Cálculo del área de una superficie}
Dada la superficie $\Sigma$ de ecuación $\vec{X}=\vec{F}(u,v)$ con $(u,v) \in D$, si $\Sigma$ es sueve y simple,
el área de la misma se puede calcular:
$$Area(D) = \integralDobleGeneral{D}{||\vec{F}'_u(u,v) \times \vec{F}'_v(u,v)||}{u}{v}$$
con $D \in R^2$ acotado y de frontera de área nula.
\subsubsection*{Ejemplazo}
Calcule el área de la superfice de ecuación $z=xy$
con $(x,y)\in D = \{ (x,y) \in R^2 : x^2+ y^2 \leq 9 \}$

$$
\vec{F} (x,y) = (x,y,xy) \, (x,y) \in D
$$

$$
\vec{F}'_x(x,y) \times \vec{F}'_y(x,y)= 
\begin{vmatrix}
\hat{i} & \hat{j} & \hat{k}\\
1 & 0 & y \\
0&1&x
\end{vmatrix}
= (-y, -x, 1)
$$

$$
\therefore ||\vec{F}'_x(x,y) \times \vec{F}'_y(x,y)||
= \sqrt{y^2+x^2+1}
$$

$$
Area(\Sigma) = 
\integralDobleGeneral{D}{||\vec{F}'_x(x,y) \times \vec{F}'_y(x,y)||}{x}{y}
=
\integralDobleGeneral{D}{\sqrt{y^2+x^2+1}}{x}{y}
$$

Realizamos un cambio de variable $\begin{cases}
x = r cos(\theta) \\ 
y = r sen(\theta)
\end{cases}$

$$\sqrt{1+(rcos(\theta))^2 +(rsen(\theta))^2}$$
$$\sqrt{1+r^2cos(\theta)^2 +r^2sen(\theta)^2}$$
$$\sqrt{1+(r^2) (cos^2(\theta) + sen^2(\theta))}$$
$$\sqrt{1 + r^2}$$
y en cuanto a los límites de integración como $D$ está definida
en $r^2cos(\theta)^2 +r^2sen(\theta)^2 \leq 9$, $r^2 \leq 9$, $r \leq 3$
por ende $0 < r \leq 3$ y $0 \leq \theta < 2\pi$.

$$
\integralDobleConLimites{2\pi}{0}{3}{0}{\sqrt{1 + r^2} r}{r}{\theta}
$$
\subsection{Integral de superficie de campo escalar}
Dado $f:H \subset R^n \rightarrow R^n$, suponiendo un conjunto continuo en $H$ (donde está incluida
la superficie $\Sigma$), el símbolo $\int \int \limits_{\Sigma} f \, d\sigma$ representa
la integral de superficie de $f$ en $\Sigma$.
\begin{itemize}
    \item Donde $f$ se evalúa en todos los puntos de $\Sigma$
    \item $d\sigma = || \vec{F}'_u(u,v) \times \vec{F}'_v(u,v)|| \, du \, dv$ es el diferencial de área de superficie,
\end{itemize}

$$
\integralDobleGeneral{D}{f(\vec{F}(u,v)) || \vec{F}'_u(u,v) \times \vec{F}'_v(u,v) ||}{u}{v}
$$
\subsection{Aplicaciones}
\begin{figure}[H] 
    \centering    
    \includegraphics[width=0.9\textwidth]{img/int_sup_ap.png}
    \label{fig:int_sup_ap}
\end{figure}
\subsubsection*{Ejemplazo}
Calcule el valor medio de $f(x,y,z) = yz$ sobre el trozo de plano $\Sigma$ de ecuación $z=3x$ con 
$D: x^2 + y^2 \leq 2y$.

$$\vec{F}(x,y)= (x, y, 3x) , (x,y) \in D$$ 
$$f(\vec{F}(x,y)) = f(x, y, 3x) = 3yx$$ 

$$
||\vec{F}'_x(x,y) \times \vec{F}'_y(x,y)||=
\begin{vmatrix}
\hat{i} & \hat{j} & \hat{k}\\
1&0&3\\
0&1&0
\end{vmatrix}
= || (-3, 0, 1) || = \sqrt{10}
$$

$$
Area(\Sigma) = 
\sqrt{10} \integralDobleGeneral{D}{}{x}{y}
$$

$$
3\sqrt{10} \integralDobleGeneral{D}{yx}{x}{y}
$$

$$
\begin{cases}
    x=rcos(\theta) \\
    y=rsen(\theta)
\end{cases}
$$

como $D: x^2+y^2 \leq 2y$, $D: r^2 \leq 2 r sen(\theta)$, $D: 0 \leq r \leq 2sen(\theta)$
Y como $x^2+(y-1)^2 \leq 1$, el centro del círculo está en $(0,1)$ y tiene $r = 1$, por ende,
$0 \leq \theta \leq \pi$



\textit{Terminar\dots}

\subsection{Integral de superficie de campo vectorial (flujo)}
Dado $\vec{f}:H \subset R^n \rightarrow R^n$, suponiendo que $\vec{f}$ es continuo en $H$
(donde está incluida la superficie $\Sigma$) $\int \int \limits_{\Sigma} \vec{f} \hat{n} \, d\sigma$
representa la integral de superficie o de flujo de $\vec{f}$ a través de $\Sigma$.

$$
\int\int\limits_{\Sigma} \vec{f} \hat{n} \, d\sigma =
\integralDobleGeneral{\Sigma}{\vec{f}(\vec{F}(u,v)). \vec{F}'_u(u,v) \times \vec{F}'_v(u,v)}{u}{v}
$$


\begin{figure}[H] 
    \centering    
    \includegraphics[width=0.4\textwidth]{img/n.png}
    \label{fig:n}
\end{figure}
