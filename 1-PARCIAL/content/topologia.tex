\section{Topología}

\begin{nota}[colback=blue!5,colframe=blue!60, title=Definiciones]
\begin{itemize}
    \item $\esferaAbierta$ conjunto de puntos cuya distancia a un punto $A$ es menor que un radio $r$.
    \item $\entornoDeA$ es todo conjunto capaz de incluir una esfera abierta con centro en el punto $A$.
    \item $\entornoReducidoDeA$ es el entorno de $A$ sin el punto $A$.
\end{itemize}
\end{nota}

\subsection*{Puntos}
Siendo $S \subset R^n$ y $A \in R^n$:
\begin{itemize}
    \item $A$ es un \textbf{punto interior} de $S$ si existe un $\entornoDeA$ contenido en $S$.
    \item $A$ es \textbf{punto exterior} a $S$ cuando existe algún $\entornoDeA$ que no tiene puntos de $S$.
    \item $A$ es \textbf{punto frontera} de $S$ cuando en todo $\entornoDeA$ 
    existe algún punto de $S$ y alguno que no pertenece a $S$.
    \item Un punto $A \in S \subset R^n$ es \textbf{punto aislado} de $S$ cuando existe algún 
    $\entornoReducidoDeA$ que no tiene puntos de $S$.
    \item Un punto $A$ es \textbf{punto de acumulación} de $S$ cuando en todo $\entornoReducidoDeA$ existe algún punto de $S$.
\end{itemize}
\begin{figure}[H] 
    \centering    
    \includegraphics[width=1\textwidth]{img/puntos.png}
    \label{fig:puntos}
\end{figure}
\subsection*{Tipos de conjuntos de puntos}
\begin{itemize}
    \item El \textbf{interior} de $S$ es el conjunto de sus puntos interiores.
    \item El \textbf{exterior} de $S$ es el conjunto de sus puntos exteriores.
    \item La \textbf{frontera} de $S$ es el conjunto de sus puntos frontera. Denotaremos con $\partial S$ a la frontera de $S$.
\end{itemize}
\subsection*{Conjuntos de puntos}
\begin{itemize}
    \item $S \subset R^n$ es un \textbf{conjunto abierto} cuando todos sus puntos son interiores.
    \item $S \subset R^n$ es un \textbf{conjunto cerrado} cuando contiene a todos sus puntos de acumulación.
    \item $S \subset R^n$ es un \textbf{conjunto acotado} cuando se lo puede incluir en una esfera abierta con centro en el origen y
radio finito.
    \item $S \subset R^n$ es un \textbf{conjunto compacto} cuando es cerrado y acotado.
    \item $S \subset R^n$ es un \textbf{conjunto conexo} cuando dados dos puntos cualesquiera $A,B \in S$, “es posible ir
desde uno al otro desplazándose por puntos de $S$”. 
\end{itemize}

