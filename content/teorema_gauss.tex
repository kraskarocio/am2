\section{Teorema de Gauss}

\begin{figure}[H] 
    \centering    
    \includegraphics[width=0.8\textwidth]{img/gauss1.png}
    \label{fig:gauss1}
\end{figure}
\begin{figure}[H] 
    \centering    
    \includegraphics[width=0.8\textwidth]{img/gauss2.png}
    \label{fig:gauss2}
\end{figure}

\subsection{Teorema de la divergencia o de Gauss}
Sea un cuerpo $D$ , su superficie frontera $\partial D$ y un campo $\vec{f}$, si:
\begin{enumerate}
    \item $D$ es acotado con frontera $\partial D$ de volumen nulo y se puede subdividir en cantidad finita de regiones tipo $RS4$ de $R^3$.
    \item $\partial D$ es una superficie cerrada, suave a trozos, simple y orientable. Orientada en forma saliente de $D$.
    \item $\vec{f} \in C^1$ en todo punto de $D$ y $\partial D$ , o en un conjunto abierto que los incluya.
\end{enumerate}
\begin{figure}[H] 
    \centering    
    \includegraphics[width=0.5\textwidth]{img/gauss3.png}
    \label{fig:gauss3}
\end{figure}
\begin{figure}[H] 
    \centering    
    \includegraphics[width=0.3\textwidth]{img/gauss4.png}
    \label{fig:gauss3}
\end{figure}
Es decir, el flujo de $\vec{f}$ a través de la superficie frontera de $D$ es igual a la integral triple de la divergencia
de $\vec{f}$ extendida a $D$.
