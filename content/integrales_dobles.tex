\section{Integrales dobles}
Teorema de Fubini: Sea $f:H=[a,b]\times[c,d]$ una función acotada. Si existen
$$\integralDobleEnD = \int ^b_a (\int ^d_c f(x,y)\, dy) \, dx$$
\subsection{Cálculo de la integral doble en una región simple}
\subsubsection{Región simple tipo 1 (RS1)}
$$ RS1 = \{ (x,y) \in R^2 / y_1(x) \leq y \leq y_2(x) \land a \leq x \leq b\}$$

\subsubsection{Región simple tipo 2 (RS2)}
$$ RS2 = \{ (x,y) \in R^2 / x_1(y) \leq x \leq x_2(y) \land a \leq y \leq b\}$$



Una región simple es tipo 3 ($RS3$) cuando es $RS1$ y $RS2$, por ejemplo, un rectángulo o un círculo.

\subsection{Propiedades}

\begin{figure}[H] 
    \centering    
    \includegraphics[width=0.9\textwidth]{img/propiedades_int_dobles.jpg}
    \label{fig:prop_int_dobles}
\end{figure}

\subsection{Aplicaciones}

\begin{itemize}
    \item Sí $f(x,y) = 1 \, \forall(x,y) \in D$ entonces $Area(D)= \int \int \limits_{D} \, dx \, dy$.
    \item Sí $\delta (x,y)$ es la densidad superficial de masa, entonces $Masa(D) =  \int \int \limits_{D} \delta(x,y) \, dx \, dy$.
    \item Si $f(x,y)$ es cualquier función continua definida en $D$ entonces $f_{med}=\frac{1}{Area(D)} \int \int \limits_{D} f(x,y) \,dx \, dy$.
    \item Calcular el centro de masa de una chapa: 
    $X_G=\frac{M_x}{Masa(D)} = \frac{\int \int \limits_{D} x\delta(x,y) \,dx \,dy }{\int \int \limits_{D} \delta(x,y) \,dx \,dy}$, y
    $Y_G=\frac{M_y}{Masa(D)} = \frac{\int \int \limits_{D} y\delta(x,y) \,dx \,dy }{\int \int \limits_{D} \delta(x,y) \,dx \,dy}$
\end{itemize}

\subsection{Cambio de variables en integrales dobles}
Esto se utiliza cuando $D$ tiene una forma rara.
\begin{figure}[H] 
    \centering    
    \includegraphics[width=0.4\textwidth]{img/dominios.png}
    \label{fig:dominios}
\end{figure}
Dada la integral $\integralDobleEnD$ se desea calcularla
aplicando el cambio de variables definido por: $\vec{h}(u,v)=(x(u,v), y(u,v))$


Denotando $J(u,v) = det(D\vec{h}(u,v))$ al jacobiano de la transformación, si se cumple que:
\begin{enumerate}
    \item $f$ es integrable en $D$.
    \item $\vec{h} \in C^1$ en un conjunto abierto que incluya a $D^*$.
    \item $J(u,v) \neq 0$ en todo punto de $D^*$
    \item Existe $\vec{h}^{-1}$ tal que $\forall (x,y) \in D$, $\vec{h}^{-1}(x,y)=(u,v) \in D^*$
\end{enumerate}
$$
\integralDobleEnD = \integralDobleGeneral{D^*}{u}{v}{f(\vec{h}(u,v)) |J(u, v)|}
$$
Es denominada la fórmula de cambio de variables en integrales dobles.

\subsubsection{Transformación líneal}
Si $(x,y) = \vec{h}(u,v)=(au+bv, cu + dv)$ con $a,b,c,d \in R$ entonces:

$$
J(u, v)= 
\begin{vmatrix}
X'_u(u,v) & X'_v(u,v) \\
Y'_u(u,v) & Y'_v(u,v)
\end{vmatrix}
=
\begin{vmatrix}
a & b \\
c & d
\end{vmatrix}
= ad - bc \neq 0
$$

$$
\integralDobleEnD =
\integralDobleGeneral{D^*}{u}{v}{f(au+bv, cu+dv) |ad - bc|}
$$

\subsubsection{Coordenadas polares}
Dado $\begin{cases}
    x = r cos(\theta)\\
    y = r sen(\theta)
\end{cases}$ donde $0 \leq r < \infty$, $0 \leq \theta < 2\pi$, por ende 
$\vec{h}(r, \theta)=(r cos(\theta), r sen(\theta))$


$$
J(r, \theta)= 
\begin{vmatrix}
cos(\theta) & - r sen(\theta) \\
sen(\theta) & r cos(\theta)
\end{vmatrix} = r
$$

$$
\integralDobleEnD = 
\integralDobleGeneral{D^*}{r}{\theta}{f(r cos(\theta), r sen(\theta)) r}
$$